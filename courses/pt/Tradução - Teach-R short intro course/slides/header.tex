% Allowing for slides with 2 columns
\def\begincols{\begin{columns}[c]}
\def\endcols{\end{columns}}
\def\begincol{\begin{column}{0.5\textwidth}}
\def\endcol{\end{column}}



% Reducing black space between R code and output
% \setlength{\topsep}{0pt}{}
\setlength{\emergencystretch}{0em}
\setlength{\parskip}{2pt}
\setlength{\partopsep}{1pt}


%%% Reducing font size of R code and output
%%% code below from http://stackoverflow.com/a/38324868
%%% see also http://stackoverflow.com/a/39961605

% %% change fontsize of R code
% \let\oldShaded\Shaded
% \let\endoldShaded\endShaded
% \renewenvironment{Shaded}{\footnotesize\oldShaded}{\endoldShaded}
%
% %% change fontsize of output
% \let\oldverbatim\verbatim
% \let\endoldverbatim\endverbatim
% \renewenvironment{verbatim}{\footnotesize\oldverbatim}{\endoldverbatim}
% %%%


% code below taken from D. Eddelbuettel
% https://github.com/eddelbuettel/samples-rmarkdown-metropolis/blob/master/header.tex

%% If you have the Fira font installed, to actually have it used it
%% via rmarkdown you need to declare it here
% \setsansfont[ItalicFont={Fira Sans Light Italic},BoldFont={Fira Sans},BoldItalicFont={Fira Sans Italic}]{Fira Sans Light}
% \setmonofont[BoldFont={Fira Mono Medium}]{Fira Mono}
% FRS: but this seems not able to print equations well...
% \setsansfont{CabritoSemiNormRegular}
% \setmonofont{Ubuntu Mono}

%% You can set various Metropolis options via \metroset{} here
\metroset{numbering=counter, progressbar=frametitle}

% https://tex.stackexchange.com/questions/54905/how-to-modify-default-beamercolortheme

%% You can redefine colours, mostly by borrowing from Beamer
\setbeamercolor{frametitle}{bg=teal}
\setbeamertemplate{navigation symbols}{}
\setbeamercolor{background canvas}{bg=white}
\setbeamercolor{background}{bg=white}
\setbeamercolor{progress bar}{%
 use=alerted text,
 fg=orange,
 bg=teal
 }
 \setbeamercolor{title separator}{
 use=progress bar,
 parent=progress bar
 }
 \setbeamercolor{progress bar in head/foot}{%
 use=progress bar,
 parent=progress bar
 }
\setbeamercolor{normal text}{fg=black,bg=white}
\setbeamercolor{alerted text}{fg=teal}
% \setbeamercolor{example text}{fg=green!50!black}
% \setbeamercolor{structure}{fg=beamer@blendedblue}
% \setbeamercolor{background canvas}{parent=normal text}
% \setbeamercolor{background}{parent=background canvas}


%% You also use hyperref, and pick colors
\hypersetup{colorlinks,citecolor=gray,filecolor=gray,linkcolor=gray,urlcolor=brown}

%% when rendered with rmarkdown, somehow the unicode char for the dot
%% disappears so we redefine it here
\renewcommand{\textbullet}{$\cdot$}
% \renewcommand{\itemBullet}{▸}   % unicode U+25b8 'black right pointing small triangle'

%% The institute macro puts a small line for affiliation at the bottom
\institute{Lab. de Ecologia Quantitativa, UFAL, Brazil \\ https://marcosvital.github.io}

%% We can also place a logo
\titlegraphic{\hfill\includegraphics[height=1.5cm]{images/TeachR.png}}
